\documentclass[PMO,authoryear,toc]{lsstdoc}
% lsstdoc documentation: https://lsst-texmf.lsst.io/lsstdoc.html
\input{meta}

% Package imports go here.

% Local commands go here.

%If you want glossaries
%% DO NOT EDIT - generated by ../lsst-texmf/bin/generateAcronyms.py from https://lsst-texmf.lsst.io/.
\newacronym{CP} {CP} {Cerro Pachon}
\newacronym{DHCP} {DHCP} {Dynamic Host Configuration Protocol}
\newacronym{DNS} {DNS} {Domain Name System}
\newglossaryentry{Docker} {name={Docker}, description={A system for packaging and distributing software using self-contained containers which may be run on any Linux system; \url{https://www.docker.com/}}}
\newacronym{GCP} {GCP} {Google Cloud Platform}
\newacronym{IT} {IT} {Information Technology}
\newacronym{ITTN} {ITTN} {IT Technical Note}
\newglossaryentry{Kubernetes} {name={Kubernetes}, description={A system for automating application deployment and management using software containers (e.g. Docker); \url{https://kubernetes.io}}}
\newacronym{LS} {LS} {La Serena}
\newacronym{RKE} {RKE} {Rancher \gls{Kubernetes} Engine}
\newacronym{RST} {RST} {reStructuredTex}
\newacronym{TU} {TU} {Tucson}
\newacronym{WUI} {WUI} {Web User Interface}

%\makeglossaries

\title{Computing Infrastructure}

% Optional subtitle
% \setDocSubtitle{A subtitle}

\author{%
Ivan Gonzalez
}

\setDocRef{ITTN-014}
\setDocUpstreamLocation{\url{https://github.com/lsst-it/ittn-014}}

\date{\vcsDate}

% Optional: name of the document's curator
% \setDocCurator{The Curator of this Document}

\setDocAbstract{%
Vera Rubin Observatory has several locations with computing infrastructure. This document details such infrastructure
}

% Change history defined here.
% Order: oldest first.
% Fields: VERSION, DATE, DESCRIPTION, OWNER NAME.
% See LPM-51 for version number policy.
\setDocChangeRecord{%
  \addtohist{1}{YYYY-MM-DD}{Unreleased.}{Ivan Gonzalez}
}


\begin{document}

% Create the title page.
\maketitle
% Frequently for a technote we do not want a title page  uncomment this to remove the title page and changelog.
% use \mkshorttitle to remove the extra pages

% ADD CONTENT HERE
% You can also use the \input command to include several content files.

\appendix
% Include all the relevant bib files.
% https://lsst-texmf.lsst.io/lsstdoc.html#bibliographies
\section{References} \label{sec:bib}
\renewcommand{\refname}{} % Suppress default Bibliography section
\bibliography{local,lsst,lsst-dm,refs_ads,refs,books}

% Make sure lsst-texmf/bin/generateAcronyms.py is in your path
\section{Acronyms} \label{sec:acronyms}
\addtocounter{table}{-1}
\begin{longtable}{p{0.145\textwidth}p{0.8\textwidth}}\hline
\textbf{Acronym} & \textbf{Description}  \\\hline

CP & Cerro Pachon \\\hline
DHCP & Dynamic Host Configuration Protocol \\\hline
DNS & Domain Name System \\\hline
GCP & Google Cloud Platform \\\hline
GKE & Google Kubernetes Engine \\\hline
IT & Information Technology \\\hline
ITTN & IT Technical Note \\\hline
LS & La Serena \\\hline
RKE & Rancher Kubernetes Engine \\\hline
RST & reStructuredTex \\\hline
TU & Tucson \\\hline
WUI & Web User Interface \\\hline
\end{longtable}

% If you want glossary uncomment below -- comment out the two lines above
%\printglossaries





\end{document}
