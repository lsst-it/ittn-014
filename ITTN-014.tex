\documentclass[PMO,authoryear,toc]{lsstdoc}
% lsstdoc documentation: https://lsst-texmf.lsst.io/lsstdoc.html
%\input{meta}

% Package imports go here.

% Local commands go here.

%If you want glossaries
%% DO NOT EDIT - generated by ../lsst-texmf/bin/generateAcronyms.py from https://lsst-texmf.lsst.io/.
\newacronym{CP} {CP} {Cerro Pachon}
\newacronym{DHCP} {DHCP} {Dynamic Host Configuration Protocol}
\newacronym{DNS} {DNS} {Domain Name System}
\newglossaryentry{Docker} {name={Docker}, description={A system for packaging and distributing software using self-contained containers which may be run on any Linux system; \url{https://www.docker.com/}}}
\newacronym{GCP} {GCP} {Google Cloud Platform}
\newacronym{IT} {IT} {Information Technology}
\newacronym{ITTN} {ITTN} {IT Technical Note}
\newglossaryentry{Kubernetes} {name={Kubernetes}, description={A system for automating application deployment and management using software containers (e.g. Docker); \url{https://kubernetes.io}}}
\newacronym{LS} {LS} {La Serena}
\newacronym{RKE} {RKE} {Rancher \gls{Kubernetes} Engine}
\newacronym{RST} {RST} {reStructuredTex}
\newacronym{TU} {TU} {Tucson}
\newacronym{WUI} {WUI} {Web User Interface}

%\makeglossaries

\title{Computing Infrastructure}

% Optional subtitle
% \setDocSubtitle{A subtitle}

\author{%
Ivan Gonzalez
}

\setDocRef{ITTN-014}
\setDocUpstreamLocation{\url{https://github.com/lsst-it/ittn-014}}

\date{\today}

% Optional: name of the document's curator
% \setDocCurator{The Curator of this Document}

\setDocAbstract{%
The Vera Rubin Observatory Computer Infrastructure is formed by four sites: La Serena, Cerro Pachon, Tucson and Cloud, each organized in different clusters. This document details the Cluster's amount of Nodes, Processing Capacity (CPU), Network Interfaces Information and Memory (RAM) Infrastructure per site.
}

% Change history defined here.
% Order: oldest first.
% Fields: VERSION, DATE, DESCRIPTION, OWNER NAME.
% See LPM-51 for version number policy.
\setDocChangeRecord{%
  \addtohist{1}{2020-04-30}{LSST RST Version}{Ivan Gonzalez}
  \addtohist{2}{2021-05-11}{lsstdoc LaTeX Version}{Ivan Gonzalez}
  \addtohist{3}{2021-06-04}{First Revision Concluded}{Ivan Gonzalez}
  \addtohist{4}{2021-09-23}{Disk Storage added}{Ivan Gonzalez}
}


\begin{document}

% Create the title page.
\maketitle
% Frequently for a technote we do not want a title page  uncomment this to remove the title page and changelog.
% use \mkshorttitle to remove the extra pages

% ADD CONTENT HERE
% You can also use the \input command to include several content files.
\section{Chile}
\subsection{La Serena}
For details of La Serena infrastructure please visit https://confluence.lsstcorp.org/x/4APMBw
\subsubsection{Antu}
\begin{itemize}
  \itemsep0em 
  \item Name:      Commissioning Cluster
  \item Platform:  Kubernetes
  \item Multicast: Yes
  \item Apps:      Nublado
  \item Nodes:     16
  \item CPU:       880
  \item Memory:    2.7TB
\end{itemize}
\begin{center}
  \small
  \begin{tabular}{||c c c c||} 
    \hline
    Node & CPU & Memory & Network \\ [0.5ex]
    \hline
    antu01.ls.lsst.org & 72 & 376GB & 2x10GB \\
    \hline
    antu02.ls.lsst.org & 72 & 376GB & 2x10GB \\
    \hline
    antu03.ls.lsst.org & 72 & 376GB & 2x10GB \\
    \hline
    antu04.ls.lsst.org & 64 & 187GB & 2x10GB \\
    \hline
    antu05.ls.lsst.org & 64 & 187GB & 2x10GB \\
    \hline
    antu06.ls.lsst.org & 64 & 187GB & 2x10GB \\
    \hline
    antu07.ls.lsst.org & 64 & 187GB & 2x10GB \\
    \hline
    antu08.ls.lsst.org & 64 & 187GB & 2x10GB \\
    \hline
    antu09.ls.lsst.org & 64 & 187GB & 2x10GB \\
    \hline
    ls-dtn01.ls.lsst.org & 48 & 187GB & 2x10GB \\
    \hline
    ls-dtn02.ls.lsst.org & 48 & 187GB & 2x10GB \\
    \hline
    ls-efdetl01.ls.lsst.org & 40 & 187GB & 2x10GB \\
    \hline
    ls-fwrd01.ls.lsst.org & 48 & 92.8GB & 2x10GB \\
    \hline
    ls-fwrd02.ls.lsst.org & 48 & 92.8GB & 2x10GB \\
    \hline
    ls-handoff01.ls.lsst.org & 24 & 187GB & 2x10GB \\
    \hline
    ls-handoff02.ls.lsst.org & 24 & 187GB & 2x10GB \\
    \hline
  \end{tabular}
\end{center}

\newpage
\subsubsection{Core LS}
\begin{itemize}
  \itemsep0em 
  \item Name:      IT Services
  \item Platform:  Libvirt
  \item Multicast: No
  \item Apps:      Foreman, DHCP and DNS
  \item Nodes:     3
  \item CPU:       96
  \item Memory:    240GB
\end{itemize}
\begin{center}
  \small
  \begin{tabular}{||c c c c||} 
    \hline
    Node & CPU & Memory & Network \\ [0.5ex]
    \hline
    core01.ls.lsst.org & 32 & 80GB & 2x1GB \\
    \hline
    core02.ls.lsst.org & 32 & 80GB & 2x1GB \\
    \hline
    core03.ls.lsst.org & 32 & 80GB & 2x1GB \\
    \hline
  \end{tabular}
\end{center}

\subsubsection{Core Dev}
\begin{itemize}
  \itemsep0em 
  \item Name:      IT Development
  \item Platform:  Libvirt
  \item Multicast: No
  \item Apps:      Foreman, DHCP and DNS
  \item Nodes:     2
  \item CPU:       36
  \item Memory:    96GB
\end{itemize}
\begin{center}
  \small
  \begin{tabular}{||c c c c||}
    \hline
    Node & CPU & Memory & Network \\ [0.5ex]
    \hline
    core1.dev.lsst.org & 32 & 80GB & 1x1GB \\
    \hline
    core2.dev.lsst.org & 04 & 16GB & 1x1GB \\
    \hline
  \end{tabular}
\end{center}

\newpage
\subsubsection{Kueyen}
\begin{itemize}
  \itemsep0em 
  \item Name:      Kueyen Cluster
  \item Platform:  Kubernetes
  \item Multicast: Yes
  \item Apps:      CSC and EFD
  \item Nodes:     4
  \item CPU:       128
  \item Memory:    312GB
\end{itemize}
\begin{center}
  \small
  \begin{tabular}{||c c c c||}
    \hline
    Node & CPU & Memory & Network \\ [0.5ex]
    \hline
    kueyen01.ls.lsst.org & 32 & 77.9GB & 2x10GB \\
    \hline
    kueyen02.ls.lsst.org & 32 & 77.9GB & 2x10GB \\
    \hline
    kueyen03.ls.lsst.org & 32 & 77.9GB & 2x10GB \\
    \hline
    kueyen04.ls.lsst.org & 32 & 77.9GB & 2x10GB \\
    \hline
  \end{tabular}
\end{center}

\subsubsection{Oracle}
\begin{itemize}
  \itemsep0em 
  \item Name:      Oracle Nodes
  \item Apps:      Oracle
  \item Nodes:     2
  \item CPU:       96
  \item Memory:    384GB
\end{itemize}
\begin{center}
  \small
  \begin{tabular}{||c c c c||} 
    \hline
    Node & CPU & Memory & Network \\ [0.5ex] 
    \hline
    oradb01.ls.lsst.org & 48 & 192GB & 2x10GB \\
    \hline
    oradb02.ls.lsst.org & 48 & 192GB & 2x10GB \\
    \hline
  \end{tabular}
\end{center}

\newpage
\subsubsection{Ruka}
\begin{itemize}
  \itemsep0em 
  \item Name:      Ruka Cluster
  \item Platform:  Kubernetes
  \item Multicast: No
  \item Apps:      RKE
  \item Nodes:     6
  \item CPU:       256
  \item Memory:    624GB
\end{itemize}
\begin{center}
  \small
  \begin{tabular}{||c c c c||}
    \hline
    Node & CPU & Memory & Network \\ [0.5ex]
    \hline
    ruka01.ls.lsst.org & 32 & 77.9GB & 2x10GB \\
    \hline
    ruka02.ls.lsst.org & 32 & 77.9GB & 2x10GB \\
    \hline
    ruka03.ls.lsst.org & 32 & 77.9GB & 2x10GB \\
    \hline
    ruka04.ls.lsst.org & 32 & 77.9GB & 2x10GB \\
    \hline
    ruka05.ls.lsst.org & 32 & 77.9GB & 2x10GB \\
    \hline
    ruka06.ls.lsst.org & 32 & 77.9GB & 2x10GB \\
    \hline
  \end{tabular}
\end{center}

\subsubsection{Rancher LS}
\begin{itemize}
  \itemsep0em 
  \item Name:      Rancher LS
  \item Platform:  Kubernetes
  \item Multicast: No
  \item Apps:      Rancher WUI
  \item Nodes:     3
  \item CPU:       12
  \item Memory:    24GB
\end{itemize}
\begin{center}
  \small
  \begin{tabular}{||c c c c c||}
    \hline
    Node & CPU & Memory & Network & Host \\ [0.5ex]
    \hline
    rancher01.ls.lsst.org & 4 & 8GB & 1x1GB & core01.ls.lsst.org \\
    \hline
    rancher02.ls.lsst.org & 4 & 8GB & 1x1GB & core02.ls.lsst.org \\
    \hline
    rancher03.ls.lsst.org & 4 & 8GB & 1x1GB & core03.ls.lsst.org \\
    \hline
  \end{tabular}
\end{center}
\newpage
\subsection{Cerro Pachon}
\subsubsection{Andes}
\begin{itemize}
  \itemsep0em 
  \item \textbf{Name:}       Andes Cluster
  \item \textbf{Platform:}   Kubernetes
  \item \textbf{Multicast:}  Yes
  \item \textbf{Apps:}       RKE
  \item \textbf{Nodes:}      16
  \item \textbf{CPU:}        192
  \item \textbf{RAM Memory:} 1.3TB
\end{itemize}
\begin{center}
  \small
  \begin{tabular}{||c c c c||} 
    \hline
    \textbf{Node} & \textbf{CPU} & \textbf{Memory} & \textbf{Network} \\ [0.5ex]
    \hline
    andes01.cp.lsst.org & 32 & 376GB & 2x10GB \\
    \hline
    andes02.cp.lsst.org & 32 & 376GB & 2x10GB \\
    \hline
    andes03.cp.lsst.org & 32 & 345GB & 2x10GB \\
    \hline
    andes04.cp.lsst.org & 32 & 77GB & 2x10GB \\
    \hline
    andes05.cp.lsst.org & 32 & 77GB & 2x10GB \\
    \hline
    andes06.cp.lsst.org & 32 & 77GB & 2x10GB \\
    \hline
  \end{tabular}
\end{center}

\newpage
\subsubsection{Amor}
\begin{itemize}
  \itemsep0em 
  \item \textbf{Name:}       Amor Cluster
  \item \textbf{Multicast:}  Yes
  \item \textbf{Apps:}       Docker
  \item \textbf{Nodes:}      2
  \item \textbf{CPU:}        32
  \item \textbf{RAM Memory:} 128GB
\end{itemize}
\begin{center}
  \small
  \begin{tabular}{||c c c c||}
    \hline
    \textbf{Node} & \textbf{CPU} & \textbf{Memory} & \textbf{Network} \\ [0.5ex]
    \hline
    amor01.cp.lsst.org & 16 & 62GB & 2x10GB \\
    \hline
    amor02.cp.lsst.org & 16 & 62GB & 2x10GB \\
    \hline
  \end{tabular}
\end{center}

\subsubsection{Azar}
\begin{itemize}
  \itemsep0em 
  \item \textbf{Name:}       Azar Cluster
  \item \textbf{Multicast:}  Yes
  \item \textbf{Apps:}       Docker
  \item \textbf{Nodes:}      2
  \item \textbf{CPU:}        112
  \item \textbf{RAM Memory:} 752GB
\end{itemize}
\begin{center}
  \small
  \begin{tabular}{||c c c c||}
    \hline
    \textbf{Node} & \textbf{CPU} & \textbf{Memory} & \textbf{Network} \\ [0.5ex]
    \hline
    azar1.cp.lsst.org & 56 & 376GB & 2x10GB \\
    \hline
    azar2.cp.lsst.org & 56 & 376GB & 2x10GB \\
    \hline
  \end{tabular}
\end{center}

\newpage
\subsubsection{Core CP}
\begin{itemize}
  \itemsep0em 
  \item \textbf{Name:}       Summit Core Cluster
  \item \textbf{Platform:}   Libvirt
  \item \textbf{Multicast:}  No
  \item \textbf{Apps:}       Foreman, DHCP, DNS
  \item \textbf{Nodes:}      3
  \item \textbf{CPU:}        96
  \item \textbf{RAM Memory:} 231GB
\end{itemize}
\begin{center}
  \small
  \begin{tabular}{||c c c c||}
    \hline
    \textbf{Node} & \textbf{CPU} & \textbf{Memory} & \textbf{Network} \\ [0.5ex]
    \hline
    core01.cp.lsst.org & 32 & 77GB & 2x10GB \\
    \hline
    core02.cp.lsst.org & 32 & 77GB & 2x10GB \\
    \hline
    core03.cp.lsst.org & 32 & 77GB & 2x10GB \\
    \hline
  \end{tabular}
\end{center}

\subsubsection{Rancher CP}
\begin{itemize}
  \itemsep0em 
  \item \textbf{Name:}       Rancher CP
  \item \textbf{Platform:}   Kubernetes
  \item \textbf{Multicast:}  No
  \item \textbf{Apps:}       Rancher WUI
  \item \textbf{Nodes:}      3
  \item \textbf{CPU:}        12
  \item \textbf{RAM Memory:} 21GB
\end{itemize}
\begin{center}
  \small
  \begin{tabular}{||c c c c c||}
    \hline
    \textbf{Node} & \textbf{CPU} & \textbf{Memory} & \textbf{Network} & \textbf{Host} \\ [0.5ex]
    \hline
    rancher01.cp.lsst.org & 4 & 7GB & 1x1GB & core1.cp.lsst.org \\
    \hline
    rancher02.cp.lsst.org & 4 & 7GB & 1x1GB & core2.cp.lsst.org \\
    \hline
    rancher03.cp.lsst.org & 4 & 7GB & 1x1GB & core3.cp.lsst.org \\
    \hline
  \end{tabular}
\end{center}
\newpage
\section{United States}
\vspace*{-\baselineskip}
Tucson site is undergoing a rebuilding process, therefore information may not be up to date.
\subsection{Tucson Lab}
\subsubsection{Pillan}
\begin{itemize}
  \itemsep0em 
  \item \textbf{Name:}       Tucson Test Stand (TTS)
  \item \textbf{Platform:}   Kubernetes
  \item \textbf{Multicast:}  Yes
  \item \textbf{Apps:}       Nublado, Telescope Control System
  \item \textbf{System Details:} \href{https://rancher.tu.lsst.org}{Rancher TU}  

\end{itemize}

\subsubsection{Core TU}
\begin{itemize}
  \itemsep0em 
  \item \textbf{Name:}       Tucson Core Cluster
  \item \textbf{Platform:}   Libvirt
  \item \textbf{Multicast:}  No
  \item \textbf{Apps:}       Foreman, DHCP, DNS
  \item \textbf{Nodes:}      3
  \item \textbf{CPU:}        96
  \item \textbf{RAM Memory:} 231GB
  \item \textbf{Usable Storage:} X.XTB
\end{itemize}
\begin{center}
  \small
  \begin{tabular}{||c c c c c c||}
    \hline
    \textbf{Node} & \textbf{CPU} & \textbf{Memory} & \textbf{Network} & \textbf{System Storage} & \textbf{VM Storage} \\ [0.5ex]
    \hline
    core1.tu.lsst.org & 32 & 77GB & 2x10GB & 931GB & 10.9TB + 477GB \\
    \hline
    core2.tu.lsst.org & 32 & 77GB & 2x10GB & 931GB & 500GB \\
    \hline
    core3.tu.lsst.org & 32 & 77GB & 2x10GB & 931GB & 500GB \\
    \hline
  \end{tabular}
\end{center}
\newpage
\section{Cloud}
\subsection{Google Cloud Platform}
IT uses Google Cloud services to deploy development pods
\begin{itemize}
  \itemsep0em 
  \item Name:      Cloud Core Cluster
  \item Platform:  GKE
  \item Multicast: No
  \item Apps:      Graylog
  \item Nodes:     3
  \item CPU:       2
  \item Memory:    22.5GB
\end{itemize}
\begin{center}
  \small
  \begin{tabular}{||c c c c||}
    \hline
    Node & CPU & Memory & Network \\ [0.5ex]
    \hline
    node1 & 2 & 5.9GB & 1x10GB \\
    \hline
    node2 & 2 & 5.9GB & 1x10GB \\
    \hline
    node3 & 2 & 5.9GB & 1x10GB \\
    \hline
  \end{tabular}
\end{center}

\appendix
% Include all the relevant bib files.
% https://lsst-texmf.lsst.io/lsstdoc.html#bibliographies
% \section{References} \label{sec:bib}
\renewcommand{\refname}{} % Suppress default Bibliography section
\bibliography{local,lsst,lsst-dm,refs_ads,refs,books}

% Make sure lsst-texmf/bin/generateAcronyms.py is in your path
\newpage
\section{Acronyms} \label{sec:acronyms}
\addtocounter{table}{-1}
\begin{longtable}{p{0.145\textwidth}p{0.8\textwidth}}\hline
\textbf{Acronym} & \textbf{Description}  \\\hline

CP & Cerro Pachon \\\hline
DHCP & Dynamic Host Configuration Protocol \\\hline
DNS & Domain Name System \\\hline
GCP & Google Cloud Platform \\\hline
GKE & Google Kubernetes Engine \\\hline
IT & Information Technology \\\hline
ITTN & IT Technical Note \\\hline
LS & La Serena \\\hline
RKE & Rancher Kubernetes Engine \\\hline
RST & reStructuredTex \\\hline
TU & Tucson \\\hline
WUI & Web User Interface \\\hline
\end{longtable}

% If you want glossary uncomment below -- comment out the two lines above
%\printglossaries
\end{document}
