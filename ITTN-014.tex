\documentclass[PMO,authoryear,toc]{lsstdoc}
\usepackage{ragged2e}
% lsstdoc documentation: https://lsst-texmf.lsst.io/lsstdoc.html
%\input{meta}

% Package imports go here.

% Local commands go here.

%If you want glossaries
%% DO NOT EDIT - generated by ../lsst-texmf/bin/generateAcronyms.py from https://lsst-texmf.lsst.io/.
\newacronym{CP} {CP} {Cerro Pachon}
\newacronym{DHCP} {DHCP} {Dynamic Host Configuration Protocol}
\newacronym{DNS} {DNS} {Domain Name System}
\newglossaryentry{Docker} {name={Docker}, description={A system for packaging and distributing software using self-contained containers which may be run on any Linux system; \url{https://www.docker.com/}}}
\newacronym{GCP} {GCP} {Google Cloud Platform}
\newacronym{IT} {IT} {Information Technology}
\newacronym{ITTN} {ITTN} {IT Technical Note}
\newglossaryentry{Kubernetes} {name={Kubernetes}, description={A system for automating application deployment and management using software containers (e.g. Docker); \url{https://kubernetes.io}}}
\newacronym{LS} {LS} {La Serena}
\newacronym{RKE} {RKE} {Rancher \gls{Kubernetes} Engine}
\newacronym{RST} {RST} {reStructuredTex}
\newacronym{TU} {TU} {Tucson}
\newacronym{WUI} {WUI} {Web User Interface}

%\makeglossaries

\title{Computing Infrastructure}

% Optional subtitle
% \setDocSubtitle{A subtitle}

\author{%
Ivan Gonzalez
}

\setDocRef{ITTN-014}
\setDocUpstreamLocation{\url{https://github.com/lsst-it/ittn-014}}

\date{\today}

% Optional: name of the document's curator
% \setDocCurator{The Curator of this Document}

\setDocAbstract{%
The Vera Rubin Observatory Computer Infrastructure is formed by four sites: La Serena, Cerro Pachon, Tucson and Cloud, each composed of several server's clusters. This document details each Cluster's amount of Nodes, Processing Capacity (CPU), Network Interfaces Information and Memory (RAM) Infrastructure per site.
\linebreak\linebreak The Physical Storage information from each cluster will be devided into three:
\begin{itemize}
  \itemsep0em 
  \item \textbf{System Storage:} Represents the Disk Storage capacity that the OS has control of - /, /etc, /usr, /lib, /home, /var, /root, /opt-.
  \item \textbf{Data Storage:} Disk unit space dedicated for data storage or common pool contribution.
  \item \textbf{Docker Storage:} Storage Unit dedicated for running docker images - /var/lib/docker/-.
  \item \textbf{Usable Kubernetes Storage}: All Rubin Clusters are provisioned with rook-ceph. Rook orchestrates ceph storage between the distributed nodes, allowing self-managing, self-scaling, and self healing storage services. The Usable Kubernetes Storage is then the total amount of usable space for the PV's (over kubernetes).
\end{itemize}
}

% Change history defined here.
% Order: oldest first.
% Fields: VERSION, DATE, DESCRIPTION, OWNER NAME.
% See LPM-51 for version number policy.
\setDocChangeRecord{%
  \addtohist{1}{2020-04-30}{LSST RST Version}{Ivan Gonzalez}
  \addtohist{2}{2021-05-11}{lsstdoc LaTeX Version}{Ivan Gonzalez}
  \addtohist{3}{2021-06-04}{First Revision Concluded}{Ivan Gonzalez}
  \addtohist{4}{2021-09-23}{Disk Storage added}{Ivan Gonzalez}
  \addtohist{5}{2021-10-18}{Corrections to Disk Storage Information}{Ivan Gonzalez}
  \addtohist{6}{2021-11-22}{Yagan Cluster Added}{Ivan Gonzalez}
  \addtohist{7}{2022-02-25}{General Update and add Luan Cluster}{Heinrich Reinking}
}


\begin{document}

% Create the title page.
\maketitle
% Frequently for a technote we do not want a title page  uncomment this to remove the title page and changelog.
% use \mkshorttitle to remove the extra pages

% ADD CONTENT HERE
% You can also use the \input command to include several content files.
\section{Chile}
\subsection{La Serena}
For details of La Serena infrastructure please visit https://confluence.lsstcorp.org/x/4APMBw
\subsubsection{Antu}
\begin{itemize}
  \itemsep0em 
  \item \textbf{Name:}      Commissioning Cluster
  \item \textbf{Platform:}  Kubernetes
  \item \textbf{Multicast:} Yes
  \item \textbf{Apps:}      Nublado
  \item \textbf{Nodes:}     15
  \item \textbf{CPU:}       832
  \item \textbf{Memory:}    3.2TB
\end{itemize}
\begin{center}
  \small
  \begin{tabular}{||c c c c||} 
    \hline
    \textbf{Node} & \textbf{CPU} & \textbf{Memory} & \textbf{Network} \\ [0.5ex]
    \hline
    antu01.ls.lsst.org & 72 & 376GB & 2x10GB \\
    \hline
    antu02.ls.lsst.org & 72 & 376GB & 2x10GB \\
    \hline
    antu03.ls.lsst.org & 72 & 376GB & 2x10GB \\
    \hline
    antu04.ls.lsst.org & 64 & 187GB & 2x10GB \\
    \hline
    antu05.ls.lsst.org & 64 & 187GB & 2x10GB \\
    \hline
    antu06.ls.lsst.org & 64 & 187GB & 2x10GB \\
    \hline
    antu07.ls.lsst.org & 64 & 187GB & 2x10GB \\
    \hline
    antu08.ls.lsst.org & 64 & 187GB & 2x10GB \\
    \hline
    antu09.ls.lsst.org & 64 & 187GB & 2x10GB \\
    \hline
    ls-dtn01.ls.lsst.org & 48 & 187GB & 2x10GB \\
    \hline
    ls-dtn02.ls.lsst.org & 48 & 187GB & 2x10GB \\
    \hline
    ls-efdetl01.ls.lsst.org & 40 & 187GB & 2x10GB \\
    \hline
    ls-fwrd01.ls.lsst.org & 48 & 92GB & 2x10GB \\
    \hline
    ls-handoff01.ls.lsst.org & 24 & 187GB & 2x10GB \\
    \hline
    ls-handoff02.ls.lsst.org & 24 & 187GB & 2x10GB \\
    \hline
  \end{tabular}
\end{center}

\newpage
\subsubsection{Core LS}
\begin{itemize}
  \itemsep0em 
  \item \textbf{Name:}      IT Services
  \item \textbf{Platform:}        Libvirt
  \item \textbf{Multicast:}  No
  \item \textbf{Apps:}       Foreman, DHCP and DNS
  \item \textbf{Nodes:}      3
  \item \textbf{CPU:}        96
  \item \textbf{RAM Memory:} 231GB
\end{itemize}
\begin{center}
  \small
  \begin{tabular}{||c c c c||} 
    \hline
    \textbf{Node} & \textbf{CPU} & \textbf{Memory} & \textbf{Network} \\ [0.5ex]
    \hline
    core01.ls.lsst.org & 32 & 77GB & 2x1GB \\
    \hline
    core02.ls.lsst.org & 32 & 77GB & 2x1GB \\
    \hline
    core03.ls.lsst.org & 32 & 77GB & 2x1GB \\
    \hline
  \end{tabular}
\end{center}

\subsubsection{Core Dev}
\begin{itemize}
  \itemsep0em 
  \item \textbf{Name:}       IT Development
  \item \textbf{Platform:}   Libvirt
  \item \textbf{Multicast:}  No
  \item \textbf{Apps:}       Foreman, DHCP and DNS
  \item \textbf{Nodes:}      2
  \item \textbf{CPU:}        36
  \item \textbf{RAM Memory:} 96GB
\end{itemize}
\begin{center}
  \small
  \begin{tabular}{||c c c c||}
    \hline
    \textbf{Node} & \textbf{CPU} & \textbf{Memory} & \textbf{Network} \\ [0.5ex]
    \hline
    core1.dev.lsst.org & 32 & 80GB & 1x1GB \\
    \hline
    core2.dev.lsst.org & 04 & 16GB & 1x1GB \\
    \hline
  \end{tabular}
\end{center}

\newpage
\subsubsection{Kueyen}
\begin{itemize}
  \itemsep0em 
  \item \textbf{Name:}       Kueyen Cluster
  \item \textbf{Platform:}   Kubernetes
  \item \textbf{Multicast:}  Yes
  \item \textbf{Apps:}       CSC and EFD
  \item \textbf{Nodes:}      4
  \item \textbf{CPU:}        128
  \item \textbf{RAM Memory:} 312GB
\end{itemize}
\begin{center}
  \small
  \begin{tabular}{||c c c c||}
    \hline
    \textbf{Node} & \textbf{CPU} & \textbf{Memory} & \textbf{Network} \\ [0.5ex]
    \hline
    kueyen01.ls.lsst.org & 32 & 77GB & 2x10GB \\
    \hline
    kueyen02.ls.lsst.org & 32 & 77GB & 2x10GB \\
    \hline
    kueyen03.ls.lsst.org & 32 & 77GB & 2x10GB \\
    \hline
    kueyen04.ls.lsst.org & 32 & 77GB & 2x10GB \\
    \hline
  \end{tabular}
\end{center}

\subsubsection{Oracle}
\begin{itemize}
  \itemsep0em 
  \item \textbf{Name:}       Oracle Nodes
  \item \textbf{Apps:}       Oracle
  \item \textbf{Nodes:}      2
  \item \textbf{CPU:}        96
  \item \textbf{RAM Memory:} 374GB
\end{itemize}
\begin{center}
  \small
  \begin{tabular}{||c c c c||} 
    \hline
    \textbf{Node} & \textbf{CPU} & \textbf{Memory} & \textbf{Network} \\ [0.5ex] 
    \hline
    oradb01.ls.lsst.org & 48 & 187GB & 2x10GB \\
    \hline
    oradb02.ls.lsst.org & 48 & 187GB & 2x10GB \\
    \hline
  \end{tabular}
\end{center}

\newpage
\subsubsection{Ruka}
\begin{itemize}
  \itemsep0em 
  \item \textbf{Name:}       Ruka Cluster
  \item \textbf{Platform:}   Kubernetes
  \item \textbf{Multicast:}  No
  \item \textbf{Apps:}       RKE
  \item \textbf{Nodes:}      6
  \item \textbf{CPU:}        256
  \item \textbf{RAM Memory:} 610GB
\end{itemize}
\begin{center}
  \small
  \begin{tabular}{||c c c c||}
    \hline
    \textbf{Node} & \textbf{CPU} & \textbf{Memory} & \textbf{Network} \\ [0.5ex]
    \hline
    ruka01.ls.lsst.org & 32 & 78GB & 2x10GB \\
    \hline
    ruka02.ls.lsst.org & 32 & 78GB & 2x10GB \\
    \hline
    ruka03.ls.lsst.org & 32 & 78GB & 2x10GB \\
    \hline
    ruka04.ls.lsst.org & 32 & 125GB & 2x10GB \\
    \hline
    ruka05.ls.lsst.org & 32 & 125GB & 2x10GB \\
    \hline
    ruka06.ls.lsst.org & 32 & 125GB & 2x10GB \\
    \hline
  \end{tabular}
\end{center}

\subsubsection{Rancher LS}
\begin{itemize}
  \itemsep0em 
  \item \textbf{Name:}       Rancher LS
  \item \textbf{Platform:}   Kubernetes
  \item \textbf{Multicast:}  No
  \item \textbf{Apps:}       Rancher WUI
  \item \textbf{Nodes:}      3
  \item \textbf{CPU:}        12
  \item \textbf{RAM Memory:} 21GB
\end{itemize}
\begin{center}
  \small
  \begin{tabular}{||c c c c c||}
    \hline
    \textbf{Node} & \textbf{CPU} & \textbf{Memory} & \textbf{Network} & \textbf{Host} \\ [0.5ex]
    \hline
    rancher01.ls.lsst.org & 4 & 7GB & 1x1GB & core01.ls.lsst.org \\
    \hline
    rancher02.ls.lsst.org & 4 & 7GB & 1x1GB & core02.ls.lsst.org \\
    \hline
    rancher03.ls.lsst.org & 4 & 7GB & 1x1GB & core03.ls.lsst.org \\
    \hline
  \end{tabular}
\end{center}
\newpage
\subsection{Cerro Pachon}
\subsubsection{Yagan}
\begin{itemize}
  \itemsep0em 
  \item \textbf{Name:}       Rubin Production Environment
  \item \textbf{Platform:}   Kubernetes
  \item \textbf{Multicast:}  Yes
  \item \textbf{Apps:}       Nublado, Telescope Control System
  \item \textbf{System Details:} \href{https://rancher.cp.lsst.org}{Rancher CP}
\end{itemize}

\subsubsection{Yepun}
\begin{itemize}
  \itemsep0em 
  \item \textbf{Name:}       IT Production Kubernetes Cluster
  \item \textbf{Platform:}   Kubernetes
  \item \textbf{Multicast:}  Yes
  \item \textbf{Apps:}       IT Monitoring
  \item \textbf{System Details:} \href{https://rancher.cp.lsst.org}{Rancher CP}
\end{itemize}

\subsubsection{Lukay}
\begin{itemize}
  \itemsep0em 
  \item \textbf{Name:}       IT Summit S3
  \item \textbf{Platform:}   Kubernetes
  \item \textbf{Multicast:}  No
  \item \textbf{Apps:}       Rook
  \item \textbf{System Details:} \href{https://rancher.cp.lsst.org}{Rancher CP}
\end{itemize}

\subsubsection{Chonchon}
\begin{itemize}
  \itemsep0em 
  \item \textbf{Name:}       Telescope Summit S3
  \item \textbf{Platform:}   Kubernetes
  \item \textbf{Multicast:}  No
  \item \textbf{Apps:}       Rook
  \item \textbf{System Details:} \href{https://rancher.cp.lsst.org}{Rancher CP}
\end{itemize}

\subsubsection{Amor}
\begin{itemize}
  \itemsep0em 
  \item \textbf{Name:}       Love Nodes
  \item \textbf{Multicast:}  Yes
  \item \textbf{Apps:}       Docker
\end{itemize}
\begin{center}
  \small
  \begin{tabular}{||c c c c c||}
    \hline
    \textbf{Node} & \textbf{CPU} & \textbf{Memory} & \textbf{Network} & \textbf{Disk Storage} \\ [0.5ex]
    \hline
    love01.cp.lsst.org & 16 & 62GB & 2x10GB & 893GB \\
    \hline
    love02.cp.lsst.org & 16 & 62GB & 2x10GB & 893GB \\
    \hline
  \end{tabular}
\end{center}

\subsubsection{Azar}
\begin{itemize}
  \itemsep0em 
  \item \textbf{Name:}       Azar Nodes
  \item \textbf{Multicast:}  Yes
  \item \textbf{Apps:}       Docker
\end{itemize}
\begin{center}
  \small
  \begin{tabular}{||c c c c c||}
    \hline
    \textbf{Node} & \textbf{CPU} & \textbf{Memory} & \textbf{Network} & \textbf{Disk Storage} \\ [0.5ex]
    \hline
    azar1.cp.lsst.org & 56 & 376GB & 2x10GB & 11TB \\
    \hline
    azar2.cp.lsst.org & 56 & 376GB & 2x10GB & 11TB \\
    \hline
    azar3.cp.lsst.org & 56 & 256GB & 2x10GB* & 10TB \\
    \hline
  \end{tabular}
\end{center}

*: MTU 9000

\subsubsection{Core CP}
\begin{itemize}
  \itemsep0em 
  \item \textbf{Name:}       Summit Core Cluster
  \item \textbf{Platform:}   Libvirt
  \item \textbf{Multicast:}  No
  \item \textbf{Apps:}       Foreman, DHCP, DNS
  \item \textbf{Nodes:}      3
  \item \textbf{CPU:}        96
  \item \textbf{RAM Memory:} 231GB
\end{itemize}
\begin{center}
  \small
  \begin{tabular}{||c c c c c||}
    \hline
    \textbf{Node} & \textbf{CPU} & \textbf{Memory} & \textbf{Network} & \textbf{Disk Storage} \\ [0.5ex]
    \hline
    core01.cp.lsst.org & 32 & 77GB & 2x10GB & 500GB \\
    \hline
    core02.cp.lsst.org & 32 & 77GB & 2x10GB & 500GB \\
    \hline
    core03.cp.lsst.org & 32 & 77GB & 2x10GB & 500GB \\
    \hline
  \end{tabular}
\end{center}


\newpage
\section{United States}
\vspace*{-\baselineskip}
Tucson site is undergoing a rebuilding process, therefore information may not be up to date.
\subsection{Tucson Lab}
\subsubsection{Pillan}
\vspace*{-\baselineskip}
\begin{itemize}
  \itemsep0em 
  \item \textbf{Name:}       Pillan Cluster
  \item \textbf{Platform:}   Kubernetes
  \item \textbf{Multicast:}  Yes
  \item \textbf{Apps:}       RKE
  \item \textbf{Nodes:}      1
  \item \textbf{CPU:}        40
  \item \textbf{RAM Memory:} 30GB
\end{itemize}
\vspace*{-\baselineskip}
\begin{center}
  \small
  \begin{tabular}{||c c c c||}
    \hline
    \textbf{Node} & \textbf{CPU} & \textbf{Memory} & \textbf{Network} \\ [0.5ex]
    \hline
    pillan01.tu.lsst.org & 40 & 30GB & 1x10GB \\
    \hline
  \end{tabular}
\end{center}
\vspace*{-\baselineskip}
\subsubsection{Core TU}
\vspace*{-\baselineskip}
\begin{itemize}
  \itemsep0em 
  \item \textbf{Name:}       Tucson Core Cluster
  \item \textbf{Platform:}   Libvirt
  \item \textbf{Multicast:}  No
  \item \textbf{Apps:}       Foreman, DHCP, DNS
  \item \textbf{Nodes:}      3
  \item \textbf{CPU:}        96
  \item \textbf{RAM Memory:} 231GB
\end{itemize}
\vspace*{-\baselineskip}
\begin{center}
  \small
  \begin{tabular}{||c c c c||}
    \hline
    \textbf{Node} & \textbf{CPU} & \textbf{Memory} & \textbf{Network} \\ [0.5ex]
    \hline
    core1.tu.lsst.org & 32 & 77GB & 2x10GB \\
    \hline
    core2.tu.lsst.org & 32 & 77GB & 2x10GB \\
    \hline
    core3.tu.lsst.org & 32 & 77GB & 2x10GB \\
    \hline
  \end{tabular}
\end{center}
\newpage
\section{Cloud}
\subsection{Google Cloud Platform}
IT uses Google Cloud services to deploy development pods
\begin{itemize}
  \itemsep0em 
  \item \textbf{Name:}       Cloud Core Cluster
  \item \textbf{Platform:}   GKE
  \item \textbf{Multicast:}  No
  \item \textbf{Apps:}       Graylog
  \item \textbf{Nodes:}      3
  \item \textbf{CPU:}        6
  \item \textbf{RAM Memory:} 17.7GB
  \item \textbf{Usable Kubernetes Storage:} 200TB
\end{itemize}
\begin{center}
  \small
  \begin{tabular}{||c c c c c||}
    \hline
    \textbf{Node} & \textbf{CPU} & \textbf{Memory} & \textbf{Network} & \textbf{Storage} \\ [0.5ex]
    \hline
    node1 & 2 & 5.9GB & 1x10GB & 100GB \\
    \hline
    node2 & 2 & 5.9GB & 1x10GB & 100GB \\
    \hline
    node3 & 2 & 5.9GB & 1x10GB & 100GB \\
    \hline
  \end{tabular}
\end{center}
\appendix
% Include all the relevant bib files.
% https://lsst-texmf.lsst.io/lsstdoc.html#bibliographies
\renewcommand{\refname}{} % Suppress default Bibliography section
\bibliography{local,lsst,lsst-dm,refs_ads,refs,books}

% Make sure lsst-texmf/bin/generateAcronyms.py is in your path
\newpage
\section{Acronyms} \label{sec:acronyms}
\addtocounter{table}{-1}
\begin{longtable}{p{0.145\textwidth}p{0.8\textwidth}}\hline
\textbf{Acronym} & \textbf{Description}  \\\hline

CP & Cerro Pachon \\\hline
DHCP & Dynamic Host Configuration Protocol \\\hline
DNS & Domain Name System \\\hline
GCP & Google Cloud Platform \\\hline
GKE & Google Kubernetes Engine \\\hline
IT & Information Technology \\\hline
ITTN & IT Technical Note \\\hline
LS & La Serena \\\hline
RKE & Rancher Kubernetes Engine \\\hline
RST & reStructuredTex \\\hline
TU & Tucson \\\hline
WUI & Web User Interface \\\hline
\end{longtable}

% If you want glossary uncomment below -- comment out the two lines above
%\printglossaries

\end{document}
